\documentclass{beamer}\usepackage[]{graphicx}\usepackage[]{color}
%% maxwidth is the original width if it is less than linewidth
%% otherwise use linewidth (to make sure the graphics do not exceed the margin)
\makeatletter
\def\maxwidth{ %
  \ifdim\Gin@nat@width>\linewidth
    \linewidth
  \else
    \Gin@nat@width
  \fi
}
\makeatother

\definecolor{fgcolor}{rgb}{0.345, 0.345, 0.345}
\newcommand{\hlnum}[1]{\textcolor[rgb]{0.686,0.059,0.569}{#1}}%
\newcommand{\hlstr}[1]{\textcolor[rgb]{0.192,0.494,0.8}{#1}}%
\newcommand{\hlcom}[1]{\textcolor[rgb]{0.678,0.584,0.686}{\textit{#1}}}%
\newcommand{\hlopt}[1]{\textcolor[rgb]{0,0,0}{#1}}%
\newcommand{\hlstd}[1]{\textcolor[rgb]{0.345,0.345,0.345}{#1}}%
\newcommand{\hlkwa}[1]{\textcolor[rgb]{0.161,0.373,0.58}{\textbf{#1}}}%
\newcommand{\hlkwb}[1]{\textcolor[rgb]{0.69,0.353,0.396}{#1}}%
\newcommand{\hlkwc}[1]{\textcolor[rgb]{0.333,0.667,0.333}{#1}}%
\newcommand{\hlkwd}[1]{\textcolor[rgb]{0.737,0.353,0.396}{\textbf{#1}}}%
\let\hlipl\hlkwb

\usepackage{framed}
\makeatletter
\newenvironment{kframe}{%
 \def\at@end@of@kframe{}%
 \ifinner\ifhmode%
  \def\at@end@of@kframe{\end{minipage}}%
  \begin{minipage}{\columnwidth}%
 \fi\fi%
 \def\FrameCommand##1{\hskip\@totalleftmargin \hskip-\fboxsep
 \colorbox{shadecolor}{##1}\hskip-\fboxsep
     % There is no \\@totalrightmargin, so:
     \hskip-\linewidth \hskip-\@totalleftmargin \hskip\columnwidth}%
 \MakeFramed {\advance\hsize-\width
   \@totalleftmargin\z@ \linewidth\hsize
   \@setminipage}}%
 {\par\unskip\endMakeFramed%
 \at@end@of@kframe}
\makeatother

\definecolor{shadecolor}{rgb}{.97, .97, .97}
\definecolor{messagecolor}{rgb}{0, 0, 0}
\definecolor{warningcolor}{rgb}{1, 0, 1}
\definecolor{errorcolor}{rgb}{1, 0, 0}
\newenvironment{knitrout}{}{} % an empty environment to be redefined in TeX

\usepackage{alltt}
\usepackage{graphicx, epsfig, amsmath, amssymb}
\usepackage{wrapfig}
%\usepackage{sidecap}
%\usetheme{Copenhagen}
\usetheme{Boadilla}
%\usetheme{Warsaw}
%\usepackage[table]{xcolor}
%\definecolor{lightgray}{gray}{0.9}


\setbeamertemplate{footline}
{
\leavevmode
\hbox{\begin{beamercolorbox}[wd=0.5\paperwidth,ht=2.5ex,dp=1.125ex,leftskip=.3cm plus1fill,rightskip=.3cm]{author in head/foot}
\usebeamerfont{author in head/foot}\insertshortauthor
\end{beamercolorbox}%
\begin{beamercolorbox}[wd=0.5\paperwidth,ht=2.5ex,dp=1.125ex,leftskip=.3cm,rightskip=.3cm plus1fil]{title in head/foot}%
\usebeamerfont{title in head/foot}\insertshorttitle\hspace*{3em}
\insertframenumber{} / \inserttotalframenumber\hspace*{1ex}
\end{beamercolorbox}}%
\vskip0pt%
}

\makeatletter
\setbeamertemplate{navigation symbols}{}

\title[CC Title]{CC Title}
\author{Nehemias Ulloa}
\institute{\Large{Department of Statistics\\ Iowa State University}}
\date{\today}
\IfFileExists{upquote.sty}{\usepackage{upquote}}{}
\begin{document}

%%%%%%%%%%%%%%%%%%%%%%
%%%%% Title Page %%%%%
%%%%%%%%%%%%%%%%%%%%%%
\begin{frame}
\titlepage
\end{frame}


%%%%%%%%%%%%%%%%%%%%%%
%%%%% Outline %%%%%
%%%%%%%%%%%%%%%%%%%%%%
\begin{frame}
\frametitle{Outline}
\begin{itemize}
\item Intro and Motivation
\item Model
\item Recreate Dr. Phillips' Application
\item Our Application
\end{itemize}
\end{frame}


%%%%%%%%%%%%%%%%%%%%%%%%%%%%
%%%%% Intro/Motivation %%%%%
%%%%%%%%%%%%%%%%%%%%%%%%%%%%

\begin{frame}
\frametitle{Introduction}
Hypotheses:
\begin{itemize}
  \item Family researchers comparing the attitudes, behaviors, and opinions of pairs
  \item Collect Dyadic data
    \begin{itemize}
      \item Inter-individual reporting
      \item Intra-individual reporting
    \end{itemize}
\end{itemize}
\end{frame}

\begin{frame}
\frametitle{Motivation}
\begin{itemize}
  \item Difference scores used to analyze dyadic data
  \item Difference scores allow to see how well they ``fit'' together
  \item Common types: algebraic, absolute, and squared difference
\end{itemize}

\begin{align}
Z &= \beta_0 + \beta_1 (X - Y) + \epsilon \label{eq:diffscore} \\
Z &= \beta_0 + \beta_1 |X - Y| + \epsilon \label{eq:absdiffscore} \\
Z &= \beta_0 + \beta_1 (X - Y)^2 + \epsilon \label{eq:squarediffscore}
\end{align}
\end{frame}


\begin{frame}
\frametitle{Motivation}
\begin{itemize}
  \item Many methodological issues with Difference Scores
  \begin{enumerate}
    \item Difficult to identify the underlying mechanism
    \item Problems with underlying assumptions
  \end{enumerate}
\end{itemize}
\end{frame}


\begin{frame}
\frametitle{Motivation}
\begin{itemize}
  \item Any alternatives?
  \item Polynomial Regression \\ e.g.
  \begin{align}
    Z &= \beta_0 + \beta_1 X + \beta_2 Y + \beta_3 X^2 + \beta_4 Y^2 + \beta_5 XY + \epsilon
  \end{align}
  \item Take the Squared Difference and expand it
    \begin{align}
    (X-Y)^2 &= X^2 + Y^2 -2XY
  \end{align}
  \item Expand on the ideas from Simple Linear Regression
\end{itemize}
\end{frame}


%%%%%%%%%%%%%%%%%%%%%%%%%%%%
%%%%%%%%%% Model %%%%%%%%%%%
%%%%%%%%%%%%%%%%%%%%%%%%%%%%

\begin{frame}
\frametitle{The Model}
\begin{itemize}
  \item Theoretical model:
    \begin{align}
    Z &= \beta_0 + \beta_1 X + \beta_2 Y + \beta_3 X^2 + \beta_4 Y^2 + \beta_5 XY + \epsilon
  \end{align}
  \item Fitted model:
    \begin{align}
    \hat{z} &= b_{0} + b_{1}x + b_{2}y + b_3 x^2 + b_4 xy + b_5 y^2 \label{fittedmod} 
    \end{align}
  \item Fitted model in matrix notation:
\begin{align}
\hat{z} &= b_0 + {\bf d}^{'} {\bf b} + {\bf d}^{'} {\bf B}{\bf d}
\end{align}
where
\[ {\bf d} = \left[ \begin{array}{cc}
x  \\
y 
\end{array} \right]  \quad
%
{\bf b} = \left[ \begin{array}{cc}
b_1  \\
b_2 
\end{array} \right]  \quad
%
{\bf B} = \left[ \begin{array}{cc}
b_3 & b_4 /2 \\
b_4 /2 & b_5
\end{array} \right]
\]
\end{itemize}
\end{frame}


\begin{frame}[fragile]
\frametitle{Model}
We can fit a model like this in $\verb|R|$ using the $\verb|lm()|$ function:
\begin{knitrout}
\definecolor{shadecolor}{rgb}{0.969, 0.969, 0.969}\color{fgcolor}\begin{kframe}
\begin{alltt}
\hlcom{# Z - the response variable}
\hlcom{# X - the first explanatory variable}
\hlcom{# Y - the second explanatory variable}
\hlstd{QuadFit} \hlkwb{<-} \hlkwd{lm}\hlstd{(z} \hlopt{~} \hlstd{x} \hlopt{+} \hlstd{y} \hlopt{+} \hlkwd{I}\hlstd{(x}\hlopt{^}\hlnum{2}\hlstd{)} \hlopt{+} \hlkwd{I}\hlstd{(x}\hlopt{*}\hlstd{y)} \hlopt{+} \hlkwd{I}\hlstd{(y}\hlopt{^}\hlnum{2}\hlstd{),} \hlkwc{data}\hlstd{=d)}
\end{alltt}
\end{kframe}
\end{knitrout}
\end{frame}



%%%%%%%%%%%%%%%%%%%%%%%%%%%%%%%%%%%%%%%%%%
%%%%%%%%%% Stationarity Points %%%%%%%%%%%
%%%%%%%%%%%%%%%%%%%%%%%%%%%%%%%%%%%%%%%%%%

\begin{frame}
\frametitle{Stationarity Points: How \& Why?}
\begin{itemize}
  \item What are they?
  \begin{itemize}
    \item Points where the slope is zero no matter which direction you take the derivative
  \end{itemize}
  \item Values of our explanatory variables provide the ``best'' fit for the response \\
  \item How do you derive the stationary points?
  \begin{enumerate}
    \item Take the derivatives of Equation \ref{fittedmod} with respect to $x$ and $y$
    \item Set the derivatives equal to zero
    \item Solve for $x$ and $y$ in terms of {\bf b} and {\bf B} to find the stationarity points
    \item Refer to these points as $x_0$ and $y_0$
  \end{enumerate}
\end{itemize}
\end{frame}


\begin{frame}
\frametitle{Stationarity Points: How?}
\begin{itemize}
\item Take the partial derivatives with respect to $x$ and $y$
\begin{align*}
\left[ \begin{array}{cc}
\frac{dz}{dx} = b_1 + 2b_{3}x + b_{4}y \\
\frac{dz}{dy} = b_2 + b_{4}x + 2b_{5}y
\end{array} \right]
& = {\bf b} + 2 {\bf B d} \\
\end{align*}

\item Set the derivatives to zero and solve for ${\bf d_0}$
\begin{align*}
{\bf d_0} = \left[ \begin{array}{cc}
x_0  \\
y_0 
\end{array} \right]
%
& = -\frac{{\bf B}^{-1} {\bf b}}{2} \\
& = -\frac{1}{2} \frac{1}{b_5 b_3 - \frac{b_4^2}{2}}
\left[ \begin{array}{cc}
b_5 & -b_4 /2 \\
-b_4 /2 & b_3
\end{array} \right] 
\left[ \begin{array}{cc}
b_1  \\
b_2 
\end{array} \right] \\
& = -\frac{2}{4 b_5 b_3 - b_4^2}
\left[ \begin{array}{cc}
b_5 b_1 - b_2 b_4 /2 \\
-b_1 b_4 /2 + b_2 b_3
\end{array} \right] \\
& = 
\left[ \begin{array}{cc}
\frac{b_2 b_4 - 2 b_5 b_1}{4 b_5 b_3 - b_4^2} \\
\frac{b_1 b_4 + 2 b_2 b_3}{4 b_5 b_3 - b_4^2}
\end{array} \right] \\
\end{align*}

\end{itemize}
\end{frame}




\end{document}
